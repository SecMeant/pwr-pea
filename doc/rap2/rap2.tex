\documentclass[polish,polish,a4paper]{article}
\usepackage[T1]{fontenc}
\usepackage[utf8]{inputenc}
\usepackage{babel}
\usepackage{pslatex}
\usepackage{graphicx}
\usepackage{tikz}
\usepackage{pgfplots}
\usepackage{anysize}
\usepackage{pgfgantt}
\usepackage{tabularx}
\usepackage{float}
\usepackage{latexsym,amsmath}
\marginsize{2.5cm}{2.5cm}{3cm}{3cm}


\newcommand{\name}[1]{\sffamily\bfseries\scriptsize #1}

\newcommand{\frontpage}[8]{
%% #1 - nazwa kursu
%% #2 - kierunek 
%% #3 - termin 
%% #4 - temat 
%% #5 - problem
%% #6 - data

\vspace{2cm}

\begin{tabular}{|p{0.72\textwidth}|p{0.28\textwidth}|}
\hline
\multicolumn{2}{|c|}{}\\
\multicolumn{2}{|c|}{{\LARGE #1}}\\
\multicolumn{2}{|c|}{}\\
\hline
\name{Kierunek} & \name{Termin}\\
\multicolumn{1}{|c|}{\textit{#2}} & \multicolumn{1}{|c|}{\textit{#3}} \\
\hline
\name{Temat} & \name{Problem}\\
\multicolumn{1}{|c|}{\textit{#4}} & \multicolumn{1}{|c|}{\textit{#5}} \\
\hline
\name{Prowadzący} & \name{data}\\
\multicolumn{1}{|c|}{\textit{mgr inż. Radosław Idzikowski}} & \multicolumn{1}{|c|}{\textit{#6}} \\
\hline
\end{tabular}

}

\usepackage{listings}
\usepackage{xcolor} % for setting colors



% set the default code style
\lstset{ % General setup for the package
	basicstyle=\small,
	numbers=left,
	frame=tb,
	tabsize=2,
	columns=fixed,
	showstringspaces=false,
	showtabs=false,
	keepspaces,
	commentstyle=\color{red},
	keywordstyle=\color{blue}
}

\title{Sprawozdanie SPD}
\begin{document}
% #1 - nazwa kursu #2 - kierunek  #3 - termin #4 - temat #5 - problem #6 - data
\frontpage{Projektowanie Efektywnych Algorytmów}{Informatyka}{Czwartek 19:05}{Przeszukiwanie tabu oraz simulowanie wyżarzanie}{TSP}{\today}
\pagestyle{empty}
\newpage

\section{Opis problemu}

Problem komiwojażera (ang. travelling salesman problem, TSP) jest zagadnieniem optymalizacyjnym, polegającym na znalezieniu minimalnego cyklu Hamiltona w pełnym grafie ważonym. Do rozwiązania problemu tym razem autor posłużył się dwoma algorytmami aproksymacyjnymi o których szerzej w dalszej częsci.
Do implementacji rozwiązania posłużył język C++, gdyż zdaniem autora pozwala on na wygodne opisywanie wysokopoziomowej abstrakcji oraz dostępne kompilatory
potrafią generować efektywny kod. Testy przeprowadzane były na systemie Ubuntu Linux 19.10 64bit z kompilatorem clang 9.0 na procesorze Intel i5-3570K (4x3.8GHz) z 8GB RAM.

\section{Struktury danych - sposób przechowywania informacji}

\subsection{Graf}

Problem komiwojażera można zdefiniować dla różnej ilości wieszchołków, dalej nazywaną wielkością problemu oraz różnych wag poszczególnych dróg łączących te wierzchołki.
Z tego względu wymaganym mechanizmem wczytywania informacji o problemie będzie plik, który zawiera wielkość grafu oraz wagi poszczególnych wierzchołków.
Wagi podawane są w postaci macierzy, w której kolejne elementy w wierszu oddzielane są znakami białymi, a same wiersze oddzielane są znakiem nowej lini.
Tak zapisane dane należy sparsować i zapisać w postaci pozwalającej na jak najprostszy oraz najszybszy sposób odczytu, ponieważ będzie ona intensywnie wykorzystywana przy rozwiązywaniu problemu.
Autor zdecydował się na przechowywanie danych w postaci kwadratowej macierzy sąsiedztwa o rozmiarze równym ilości wierzchołków.
I do jej reprezentacji została wykorzystana pojedyncza tablica typów int64\_t o rozmiarze $N^{2}$, gdzie N to rozmiar problemu.
Poniżej znajduje się definicja klasy MSTMatrix która implementuje przechowywanie macierzy sąsiedztwa.

\lstinputlisting[caption=Struktura danych MSTMatrix,language=C++]
{mst.hpp}

Jeden drobny szczegół pozostaje nie wyjaśniony - typ Edge, który klasa MSTMatrix przyjmuje jako argument przy metodzie dodawania krawędzi do grafu.
Jej implementacja jest prosta i sprowadza się do przetrzymywania informacji numeru wierzchołka źródłowego i docelowego, oraz wagi opisywanego połączenia.
Metoda MSTMatrix::add korzysta z jej pól, aby odpowiednio wypełnić informacje we własnej macierzy sąsiedztwa.
Rozwiązanie problemu będzie zwracane jako typ Path ktory jest lekkim wrapperem na klase std::vector z standardowej biblioteki.
Jego implementacja znajduje sie poniżej.

\lstinputlisting[caption=Struktura danych Path,language=C++]
{path.hpp}

Został jeszcze jeden element który wymaga krótkiego wyjaśnienia. Jest to funkcja kosztu scieżki która zwraca koszt danej scieżki na podstawie argumentu w postaci macierzy sąsiedztwa
oraz sekwencji kolejno odwiedzanych wierzchołków.

\section{Metoda przeszukiwania tabu (ang. Tabu search)}

Przed omówieniem działania algorytmu przeszukiwania tabu wprowadzona musi zostać definicja ruchu (ang. move) oraz
sąsiedztwa danego rozwiązania (ang. local neighborhood).

\subsection{Operacja ruchu (ang. move)}
Operacją ruchu nazywamy taką operację na dwóch wierzchołkach istniejącego, poprawnego rozwiązania, która za pomocą dowolnych,
deterministycznych operacji zmieni szyk (kolejność) wierzchołków w drodze. W tym przypadku autor zdecydował się na implementację
trzech różnych strategii dokonywania opsiywanego ruchu. Są to zamiana wierzchołków (ang. swap), lustrzane odbicie rozpatrując 
kolejność występowania wierzchołków w zakresie wyznaczonym przez wierzchołki bedące argumentem opreacji (ang. reverse) (np.
dla sekwencji ABCDEF i argumentów B oraz E otrzymamy AEDCBF - kolejność wierzchołków w zakresie od B do E została odwrócona),
oraz wstawienie jednego z późniejszego w sekwencji z wierzchołków przed wcześniejszego (ang. insert). Wszystkie strategie są
łatwo konfigurowalne, oraz ich konfiguracja nie wpływa na czas wykonania programu (przekazywane są jako parametr szablonu).
Dla każdej z nich została przeprowadzona osobna analiza skuteczności. Implementacja każdej z nich znajduje na kolejnych listingach.

\par
Każda operacja zamiany zaimplementowana jest jako funktor który przekazywany jest do klasy odpowiedzialnej za rozwiązanie problemu.
Każdy z funktorów potrafi przeprowadzić nie tylko sam ruch, ale również spekulację na temat jego opłacalności. Taka operacja 
przydatna jest przy szukaniu najlepszego sąsiada bez decydowania się od razu na ruch. Przez to, że ruch operuje już na istniejącej
i poprawnej ścieżce (i generuje z niej kolejną poprawną) pozostaje problem wyznaczenia drogi początkowej od której ta metoda może
rozpocząć działanie. Autor zdecydował się na bardzo podobny zabieg i w tym przypadku. Tak jak wcześniej mamy tutaj parametr szablonu
, który determinuje sposób w który początkowa droga zostanie wylosowana. Dostępnymi sposobami generowania początkowej scieżki są:
zwyczajna droga składająca się z kolejnych wierzchołków od pierwszego do ostatniego, rozpoczęcie od wierzchołka zerowego oraz
kolejne, zachłanne dołączanie sąsiada o lokalnie najtańszym koszcie przejscia, oraz droga (pseudo)losowa. Ich implementacja jest
trywialna i autor zdecydował się nie zamieszczać ich implementacji w sprawozdaniu. Ich implementacja jest dostępna w publicznym
repozytorium.

\pagebreak
\lstinputlisting[caption=Zamiana wierzchołków (ang. swap),language=C++]
{swap_op.cc}

\pagebreak
\lstinputlisting[caption=Odbicie (ang. reverse),language=C++]
{invert_op.cc}

\pagebreak
\lstinputlisting[caption=Wstawienie (ang. insert),language=C++]
{insert_op.cc}

\subsection{Sąsiedztwo rozwiązania (ang. neighborhood)}
Sąsiadem rozwiązania nazywamy takie rozwiązanie, które można otrzymać poprzez pojedynczą operacji ruchu z udziałem dowolnych
dwóch różnych wierzchołków.

\subsection{Implementacja przeszukiwania}
Po zdefiniowaniu potrzebnych pojęć, sama implementacja algotytmu jest relatywnie prosta. Po zainicjowaniu odpowiednich struktur
danych i wytworzeniu początkowej scieżki rozpoczyna się głowna pętla algorytmu, której ilość iteracji wpada w obszar kalibracji
implementacji i nie jest jednoznaczna. Autor w tej konkretnej implementacji zdecydował sie na 10'000 interacji. W pętli następuje
przejscie po wszystkich możliwych kombinacjach dwóch wierzchołkach które będą kandydatami do bycia najlepszym sąsiadem.
Koszt hipotetycznego ruchu z użyciem danych dwóch wierzchołków zostaje zapisany jeżeli jest tańszy od aktualnie najtańszego, aby
na koniec można było wykonać najbardziej opłacalny ruch. Uważny odbiorca jest w stanie zauważyć, że podążając tym alorytmem możemy
znaleźć lokalnie najlepszego sąsiada, ale gorszego niz aktualne rozwiązanie przez co ostatecznie zboczyć na gorsze rozwiązanie.
Z tego powody dodatkowo przechowywana jest globalnie najlepsza scieżka jaka była widziana w czasie działania całego algorytmu i to
ona jest (oczywiście) aktualizowana w trakcie działania algorytmu oraz ostatcznie zwracana jako ostateczne rozwiązanie.
Gdy najlepszy kandydat ruchu został znaleziony, oraz sam ruch został wykonany, dodatkowo dwa wierzchołki, które wchodziły w ruch
zostają dodane do tablicy tabu i warunkowo (o tym więcej później) zablokowana na, w przypadku tej konkretnej implementacji 5
iteracji pętli. Ruch który znajduje się na liście tabu może zostać wykonany jeżeli oferuje "bardzo atrakcyjny" ruch. Atrakcyjność
ruchu wyznaczana jest na podstawie jego kosztu (podstawowe kryterium) i dodatkowo obarczone jest kosztem tym większym im większa
wartość w liście tabu widnieje dla danego ruchu (z każdą iteracja koszt tabu ulega dekrementacji do momentu aż ruch nie osiągnie
wartości 0 i zostanie uznany jako usunięty). Sama lista tabu zaimplementowana jest jako macierz, która posiada wartość tabu dla
połączenia każdego wierzchołka z każdym. Taka implementacja pozwala na szybkie odczytanie kosztu danego ruchu w trakcie jego
wykonywania. Pozostaje jedynie kwestia dekrementacja wartości tabu w liście. Z każdym obiegiem pętli algorytm musiałby przejsc
po wszystkich elementach i odpowiednio zmodyfikować ich wartość. Autor jednak zdecydował się nieznacznie zmienić implementacje
listy tabu i zamiast wartości tabu wpisywać tam numer interacji pętli w ktorym dane połączenie bedzię usunięte z listy tabu.
Można o tym myśleć jak o punkcie czasowym w jakim dane ruch zostanie uznany jako usunięty z listy. Dla przykładu:
Jeżeli aktualny numer iteracji głownej pętli równy jest 10 i właśnie dokonaliśmy ruchu to należy wpisac go do listy tabu z wartością
10 + 5 = 15. Gdzie wartość 5 wynika z tej konkretnej implementacji algorytmu. Wówczas funkcja pobierająca koszt tabu dla danego
ruchu polega na odjęciu od znalezionej tam wartości, aktualnej wartości iteracji pętli. Implementacja operacji na liście tabu
została załączona na listingu "Implementacja listy tabu - sztuczka optymalizacujna" (został uszczuplona o nieistotne szczegóły).
Dodatkowo, żeby zapobiec zatrzymania w loklanych minimach został dodany jescze jeden element do głównej pętli algorytmu.
Jest to licznik iteracji bez znalezienia lepszego rezultatu. Jeżeli ten licznik przekroczy, w przypadku tej implementacji wartość
300 iteracji, nowa droga powstaje przez wylosowanie nowej drogi a nie poprzez wykonanie ruchu. Impelementacja głównej pętli znajduje
się na listingu "Przeszukiwanie tabu - główna pętla algorytmu".

\pagebreak
\lstinputlisting[caption=Implementacja listy tabu - sztuczka optymalizacujna,language=C++]
{tabu_list.cc}

\pagebreak
\lstinputlisting[caption=Przeszukiwanie tabu - główna pętla algorytmu,language=C++]
{tabu_loop.cc}

\pagebreak
\section{Metoda symulowanego wyżarzania (ang. simulated annealing)}

Algorytm ten ma podłoże fizyczne i wynika z obserwacji wyżarzania w metarulgii. Implementacja algorytmu jest trywialna, prostsza niż
przeszukiwania tabu i daje realtywnie dobre wyniki. Przed przedstawieniem implementacji jednak należy wprowadzić pojęcie temperatury,
oraz jak będzie wpływała ona na przeszukiwanie rozwiązań. Najpierw jednak warto znać myśl przewodnią algorytmu.

\subsection{Zasada działania}
Metoda, tak jak poprzednik polega na operacjach ruchu, zdefiniowanych tak jak w przeszukiwaniu tabu, dlatego w razie wyjaśnienia
działania, implementacji ich rekonfiguralności oraz strategii, informację należy czerpać wówczas z wcześniejszych akapitów.
Algorytm składa się z kilku prostych kroków (które sprzecznie z intuicją autora daję relatywnie dobre rezultaty). Najpierw ustawiane
są wszystkie parametry na swoje wartości początkowe tj. temperatura, droga początkowa (wyznaczana na trzy sposobie, dokładnie tak samo
jak w przypadku przeszukiwania tabu) itp. Głownym punktem znów jest pętla wykonywana w ilości dobranej najczęściej empirycznie.
W pętli wyszukiwane jest losowe wybrany ruch z sąsiedzwa aktualnego rozwiązania i jest oceniany. Aktualizacja aktualnego stanu na ten wylosowany
następuje, gdy jego koszt jest większy lub zajdzie wymuszona aktualizacja (szerzej omówiona w dalszych akapitach). Dodatkowo zapisywana jest
globalnie nopotkana najlepsza scieżka i jest ona ostatecznie zwrócona jako wynik gdy algorytm zakończy działanie.

\subsection{Temperatura}
Temperatura jest zmienną która raz ustawiona na początku algorytmu na ustaloną wartość (często empirycznie, zależnie od implementacji,
a czasem nawet i od instancji problemu) i w trakcie działania algorytmu, z każdą iteracją zmienia się i wpływa na przebieg algorytmu,
pod postacią wymuszonych aktualizacji scieżki.
Zazwyczaj temperatura ustawiana jest na wartość dodatnią i wraz z działaniem algorytmu zmniejszana jest do momentu aż nie osiągnie
pewnego, ustalonego (znów zależnego od implementacji) pułapu. Strategie ochładzania (zmiejszania temperatury) mogą być różne. Autor tej
implementacji zdecydował się na trzy strategie. Ochładzanie liniowe, logarytmiczne oraz geometryczne. Sposób ich realizacji został przedstawiony
na listingu poniżej.

\lstinputlisting[caption=Implementacja ochładzania,language=c++]
{cooling.cc}

\subsection{Wymuszna aktualizacja}
Pomimo wylosowania mniej opłacalnej scieżki od aktualnej, możliwa jest jej wymuszona aktualizacja. Zabieg taki ma na celu zapobieganie
bezproduktywnego błądzenia w minimach lokalnych. W celu arbitrażu wyliczane jest prawdopodobieństwo wykonania właśnie wymuszonej aktualizacji.
Jeżeli po wylosowaniu decyzju z wyliczonem prawdopodobieństwem otrzymaliśmy wynik pozytywny, dokonujemy aktualizacji aktualnego rozwiązania,
dokładnie tak samo jak gdyby rozwiązanie było lepsze (tańsze). Pozostaje już tylko jedna nie wyjaśniona kwestia. Jak obliczać poruszane wcześniej
prawdopodobieństwo. Sposób jego wytworzenia przedstawione zostało na listingu poniżej.

\lstinputlisting[caption=Wyliczanie prawdopodobieństwa wymuszonej aktualizacji,language=c++]
{sa_prob.cc}

oraz implementacja głównej pętli algorytmu

\lstinputlisting[caption=Symulowane wyżarzanie - główna pętla algorytmu,language=c++]
{sa_solve.cc}

\pagebreak
\section{Eksperymenty obliczeniowe}
Czas wykonywanych operacji był mierzony na systemie opisanym w wprowadzeniu.
Do pomiaru czasu wykorzystano standardową bibliotekę języka C++, std::chrono.
Pomiar czasu wykonywany był w nanosekundach. Każdy pomiar wykonywany był 50 razy i ostatecznym wynikiem była średnia czasu z wszystkich prób.
Dodatkowo przed dokonywaniem takiego pomiaru uruchamiano 10 razy przebieg mierzonego kawałka kodu bez mierzenia jego czasu po to aby wytrenować układy przewidywania odwoływań do pamięci,
tj. branch prediction, cache prefetching.
Dodatkowo każdy z algorytmów przed pomiarem został dostrojony (za pomocą paramterów specyficznych dla każdego algorytmu opisanych we wcześniejszych
paragrafach) tak aby czas jego wykonania nie odbiegał znacznie od innych algorytmów. Konsekwencją tego są czasy wykonania bardzo zbliżone do siebie,
co jest powodem pominięcia przez autora analizy złożoności czasowej. Omówione natomiast zostały ich rezultaty, tj. porównanie wygenerowanej
scieżki z najlepszą możliwa dla danego zestawu danych.
Poniżej znajduje się kod realizujący mierzenie czasu.

\lstinputlisting[caption=Funkcje pomocnicze do liczenia czasu wykonania algorytmów,language=c++]
{timeutils.hpp}

\par
Poniżej znajdują się wyniki pomiarów zgromadzone dla różnych wariantów każdego z algorytmów.

\begin{center}

\begin{tabularx}{1.0\textwidth} {
	| >{\centering\arraybackslash}X
	| >{\centering\arraybackslash}X
	| >{\centering\arraybackslash}X
	| >{\centering\arraybackslash}X
	| >{\centering\arraybackslash}X | }
	\hline
	\multicolumn{5}{|c|}{Przeszukiwania tabu - metada zamiany (swap) - start losowy} \\
	\hline
	Wielkość problemu & Czas wykonania (ns) & Koszt znalezionej drogi & Koszt najlepszej drogi & błąd \\
	\hline
	17 & 4678691 & 2085 & 2085 & 0 \\
	\hline
	21 & 8122241 & 2707 & 2707 & 0 \\
	\hline
	24 & 11574596 & 1298 & 1272 & 26 \\
	\hline
	26 & 14488670 & 937 & 937 & 0 \\
	\hline
	29 & 19540981 & 1656 & 1610 & 46 \\
	\hline
	42 & 55657465 & 777 & 699 & 78 \\
	\hline
	58 & 140285446 & 30986 & 25395 & 5591 \\
	\hline
\end{tabularx}
\bigskip

\begin{tabularx}{1.0\textwidth} {
	| >{\centering\arraybackslash}X
	| >{\centering\arraybackslash}X
	| >{\centering\arraybackslash}X
	| >{\centering\arraybackslash}X
	| >{\centering\arraybackslash}X | }
	\hline
	\multicolumn{5}{|c|}{Przeszukiwania tabu - metada wstawiania (insert) - start losowy} \\
	\hline
	Wielkość problemu & Czas wykonania (ns) & Koszt znalezionej drogi & Koszt najlepszej drogi & błąd \\
	\hline
	17 & 5995145 & 3382 & 2085 & 1297 \\
	\hline
	21 & 10298361 & 4673 & 2707 & 1966 \\
	\hline
	24 & 14680920 & 3315 & 1272 & 2043 \\
	\hline
	26 & 18122428 & 2426 & 937 & 1489 \\
	\hline
	29 & 24541917 & 4855 & 1610 & 3245 \\
	\hline
	42 & 68625855 & 2934 & 699 & 2235 \\
	\hline
	58 & 179318868 & 113213 & 25395 & 87818 \\
	\hline
\end{tabularx}
\bigskip

\begin{tabularx}{1.0\textwidth} {
	| >{\centering\arraybackslash}X
	| >{\centering\arraybackslash}X
	| >{\centering\arraybackslash}X
	| >{\centering\arraybackslash}X
	| >{\centering\arraybackslash}X | }
	\hline
	\multicolumn{5}{|c|}{Przeszukiwania tabu - metada odwrócenia (reverse) - start losowy} \\
	\hline
	Wielkość problemu & Czas wykonania (ns) & Koszt znalezionej drogi & Koszt najlepszej drogi & błąd \\
	\hline
	17 & 5845171 & 2085 & 2085 & 0 \\
	\hline
	21 & 10171492 & 2707 & 2707 & 0 \\
	\hline
	24 & 14407409 & 1272 & 1272 & 0 \\
	\hline
	26 & 18198326 & 937 & 937 & 0 \\
	\hline
	29 & 24290646 & 1615 & 1610 & 5 \\
	\hline
	42 & 71764416 & 699 & 699 & 0 \\
	\hline
	58 & 169588431 & 25395 & 25395 & 0 \\
	\hline
\end{tabularx}
\bigskip

\begin{tabularx}{1.0\textwidth} {
	| >{\centering\arraybackslash}X
	| >{\centering\arraybackslash}X
	| >{\centering\arraybackslash}X
	| >{\centering\arraybackslash}X
	| >{\centering\arraybackslash}X | }
	\hline
	\multicolumn{5}{|c|}{Simulowane wyżarzanie - liniowa regulacja temperatury - start losowy} \\
	\hline
	Wielkość problemu & Czas wykonania (ns) & Koszt znalezionej drogi & Koszt najlepszej drogi & błąd \\
	\hline
	17 & 143268183& 2427 & 2085 & 342 \\
	\hline
	21 & 145967248& 3568 & 2707 & 861 \\
	\hline
	24 & 148627322& 1957 & 1272 & 685 \\
	\hline
	26 & 150143893& 1551 & 937 & 614 \\
	\hline
	29 & 153477613& 2508 & 1610 & 898 \\
	\hline
	42 & 163647317& 1897 & 699 & 1198 \\
	\hline
	58 & 177813088& 57551 & 25395 & 32156 \\
	\hline
\end{tabularx}
\bigskip

\begin{tabularx}{1.0\textwidth} {
	| >{\centering\arraybackslash}X
	| >{\centering\arraybackslash}X
	| >{\centering\arraybackslash}X
	| >{\centering\arraybackslash}X
	| >{\centering\arraybackslash}X | }
	\hline
	\multicolumn{5}{|c|}{Simulowane wyżarzanie - geometryczna regulacja temperatury - start losowy} \\
	\hline
	Wielkość problemu & Czas wykonania (ns) & Koszt znalezionej drogi & Koszt najlepszej drogi & błąd \\
	\hline
	17 & 150094480 & 2085  & 2085  & 0    \\
	\hline
	21 & 152922148 & 2757  & 2707  & 50   \\
	\hline
	24 & 155708966 & 1290  & 1272  & 18   \\
	\hline
	26 & 157297797 & 965   & 937   & 28   \\
	\hline
	29 & 160790359 & 1683  & 1610  & 73   \\
	\hline
	42 & 171444619 & 747   & 699   & 48   \\
	\hline
	58 & 186285346 & 28461 & 25395 & 3066 \\
	\hline
\end{tabularx}
\bigskip

\begin{tabularx}{1.0\textwidth} {
	| >{\centering\arraybackslash}X
	| >{\centering\arraybackslash}X
	| >{\centering\arraybackslash}X
	| >{\centering\arraybackslash}X
	| >{\centering\arraybackslash}X | }
	\hline
	\multicolumn{5}{|c|}{Simulowane wyżarzanie - logarytmiczna regulacja temperatury - start losowy} \\
	\hline
	Wielkość problemu & Czas wykonania (ns) & Koszt znalezionej drogi & Koszt najlepszej drogi & błąd \\
	\hline
	17 & 2090093 & 2085 & 2085 & 0 \\
	\hline
	21 & 2243627 & 2760 & 2707 & 53 \\
	\hline
	24 & 2072114 & 1330 & 1272 & 58 \\
	\hline
	26 & 2103146 & 1032 & 937 & 95 \\
	\hline
	29 & 2152118 & 1730 & 1610 & 120 \\
	\hline
	42 & 2366575 & 958 & 699 & 259 \\
	\hline
	58 & 2634296 & 27326 & 25395 & 1931 \\
	\hline
\end{tabularx}
\bigskip

\end{center}

\newpage

\newpage

\section{Wnioski}
\hspace{\parindent}
\par Przedstawione algorytmy w dużej mierze polegają na zdarzeniach losowych (pseudo losowch), jednak mimo to zdaniem autora relatywnie dobrze
radzą sobie z problemem komiwojarzera, szczególnie zwaracając uwage na prostotę ich implementacji, oraz ich losowy charakter.
To powiedziawszy autor pragnie zaznaczyć, że po etapie implementacji, nietrywialnym wyzwaniem okazała się kalibracja parametrów z jakimi
działa algorytm, tak aby wyniki były jak najlepsze, któremu zdecydowanie nie pomagała ich znaczna liczność.
Dobór tych parametrów bardzo często dawał skrajnie różne rezultaty przy nawet najmniejszych zmianach innych parametrów, a interakcje jakie zachodzą
między nimi są nie zawsze jasne.
Dobrane parametry, na których zostały przeprowadzone testy i przedstawione wyniki zdecydowanie nie mogą być uważane za najlepiej dobrane. Warto zaznaczyć
jednak, że autor przy ich dobieraniu, w celu stworzenia wiarygodnego prównania starał się dokonywać takich manipulacji parametrów, aby ich wyniki były
jak najlepsze z jednoczesnym zachowaniem porównywalnych czasów wykonania obu algorytmu, dzięki czemu była możliwa ich rzetelna komparacja.

\par Z otrzymanych wyników jasno wynika, że algorytm przeszukiwania tabu radzi sobie zdecydowanie lepiej od algorytmu simulowanego wyżarzania jednocześnie
będac tym, którego kalibracji parametrów dokonywało się zdecydowanie prościej. Warto także zauważyć, że najlepszym z sposobów wytwarzania sąsiedniego stanu
okazał się być algorytm reverse, czyli odwrócenia kolejności wszystkich wierzchłków z zakresu wyznaczanego przez dwa wierzchołki biorące udział w wywołaniu
procedury. Symulowane wyżarzanie ze względu na swój silny charakter losowości osiąga dużo słabsze rezultaty, co zgadza się z przewidywaniami autora.

\par Algorytm poszukiwania tabu oraz symulowanego wyżarzania pozwalają na wymierne oszacowanie optymalnej drogi i zależnie od wymagań może znaleźć
zastosowanie w praktyce.

\bibliographystyle{abbrv}
\bibliography{ref}
\end{document}

