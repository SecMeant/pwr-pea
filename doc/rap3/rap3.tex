\documentclass[polish,polish,a4paper]{article}
\usepackage[T1]{fontenc}
\usepackage[utf8]{inputenc}
\usepackage{babel}
\usepackage{pslatex}
\usepackage{graphicx}
\usepackage{tikz}
\usepackage{pgfplots}
\usepackage{anysize}
\usepackage{pgfgantt}
\usepackage{tabularx}
\usepackage{float}
\usepackage{latexsym,amsmath}
\marginsize{2.5cm}{2.5cm}{3cm}{3cm}
\tolerance=6000


\newcommand{\name}[1]{\sffamily\bfseries\scriptsize #1}

\newcommand{\frontpage}[8]{
%% #1 - nazwa kursu
%% #2 - kierunek 
%% #3 - termin 
%% #4 - temat 
%% #5 - problem
%% #6 - data

\vspace{2cm}

\begin{tabular}{|p{0.72\textwidth}|p{0.28\textwidth}|}
\hline
\multicolumn{2}{|c|}{}\\
\multicolumn{2}{|c|}{{\LARGE #1}}\\
\multicolumn{2}{|c|}{}\\
\hline
\name{Kierunek} & \name{Termin}\\
\multicolumn{1}{|c|}{\textit{#2}} & \multicolumn{1}{|c|}{\textit{#3}} \\
\hline
\name{Temat} & \name{Problem}\\
\multicolumn{1}{|c|}{\textit{#4}} & \multicolumn{1}{|c|}{\textit{#5}} \\
\hline
\name{Prowadzący} & \name{data}\\
\multicolumn{1}{|c|}{\textit{mgr inż. Radosław Idzikowski}} & \multicolumn{1}{|c|}{\textit{#6}} \\
\hline
\end{tabular}

}

\usepackage{listings}
\usepackage{xcolor} % for setting colors


% set the default code style
\lstset{ % General setup for the package
	basicstyle=\small,
	numbers=left,
	frame=tb,
	tabsize=2,
	columns=fixed,
	showstringspaces=false,
	showtabs=false,
	keepspaces,
	commentstyle=\color{red},
	keywordstyle=\color{blue}
}

\title{Sprawozdanie SPD}
\begin{document}
% #1 - nazwa kursu #2 - kierunek  #3 - termin #4 - temat #5 - problem #6 - data
\frontpage{Projektowanie Efektywnych Algorytmów}{Informatyka}{Czwartek 19:05}{Algorytmy metaheurystyczne na przykładzie algorytmu genetycznego oraz mrówkowego}{TSP}{\today}
\pagestyle{empty}
\newpage

\section{Opis problemu}

Problem komiwojażera (ang. travelling salesman problem, TSP) jest zagadnieniem optymalizacyjnym, polegającym na znalezieniu minimalnego cyklu Hamiltona w pełnym grafie ważonym. Do rozwiązania problemu tym razem autor posłużył się dwoma algorytmami - genetycznym (ang. genetic), oraz mrówkowym
(ang. ant colony optimization) o których szerzej w dalszej częsci.
Do implementacji rozwiązania posłużył język C++, gdyż zdaniem autora pozwala on na wygodne opisywanie wysokopoziomowej abstrakcji oraz dostępne kompilatory
potrafią generować efektywny kod. Testy przeprowadzane były na systemie Ubuntu Linux 19.10 64bit z kompilatorem clang 9.0 na procesorze Intel i5-3570K (4x3.8GHz) z 8GB RAM. Implementacja prezentowanego rozwiązania tworzona była z myśla o uzyskaniu jak najlepszej jej wydajności co pozwoli na wykonaniu
większej ilości iteracji algorytmów z zachowaniem tego samego czasu wykonania.
Zdaniem autora poniższa implementacja w pewnym zakresie spełnia opisane oczekiwania, chociaż bez wątpliwości jej możliwości nie zostały w pełni
wykorzystane. Szczegóły optymalizacji omawiane są w punktach poświęconych opisowi implementacji poszczególnych rozwiązań.

\section{Struktury danych - sposób przechowywania informacji}

\subsection{Graf}

Problem komiwojażera można zdefiniować dla różnej ilości wieszchołków, dalej nazywaną wielkością problemu oraz różnych wag poszczególnych dróg łączących te wierzchołki.
Z tego względu wymaganym mechanizmem wczytywania informacji o problemie będzie plik, który zawiera wielkość grafu oraz wagi poszczególnych wierzchołków.
Wagi podawane są w postaci macierzy, w której kolejne elementy w wierszu oddzielane są znakami białymi, a same wiersze oddzielane są znakiem nowej lini.
Tak zapisane dane należy sparsować i zapisać w postaci pozwalającej na jak najprostszy oraz najszybszy sposób odczytu, ponieważ będzie ona intensywnie wykorzystywana przy rozwiązywaniu problemu.
Autor zdecydował się na przechowywanie danych w postaci kwadratowej macierzy sąsiedztwa o rozmiarze równym ilości wierzchołków.
I do jej reprezentacji została wykorzystana pojedyncza tablica typów int64\_t o rozmiarze $N^{2}$, gdzie N to rozmiar problemu.
Poniżej znajduje się definicja klasy MSTMatrix która implementuje przechowywanie macierzy sąsiedztwa.

\lstinputlisting[caption=Struktura danych MSTMatrix,language=C++]
{mst.hpp}

Jeden drobny szczegół pozostaje nie wyjaśniony - typ Edge, który klasa MSTMatrix przyjmuje jako argument przy metodzie dodawania krawędzi do grafu.
Jej implementacja jest prosta i sprowadza się do przetrzymywania informacji numeru wierzchołka źródłowego i docelowego, oraz wagi opisywanego połączenia.
Metoda MSTMatrix::add korzysta z jej pól, aby odpowiednio wypełnić informacje we własnej macierzy sąsiedztwa.
Rozwiązanie problemu będzie zwracane jako typ Path ktory jest lekkim wrapperem na klase std::vector z standardowej biblioteki.
Jego implementacja znajduje sie poniżej.

\lstinputlisting[caption=Struktura danych Path,language=C++]
{path.hpp}

Został jeszcze jeden element który wymaga krótkiego wyjaśnienia. Jest to funkcja cost która zwraca koszt danej scieżki na podstawie argumentu w postaci macierzy sąsiedztwa
oraz sekwencji kolejno odwiedzanych wierzchołków.

\section{Algorytm genetyczny}
\par Algorytmy genetyczne wykorzystują towarzyszące wszystkim żywym ogrnizmom mechanizmy dotychczas dobrze zbadane i opisane w dziedzinie biologii,
Mowa tutaj o ewoulcji biologicznej gatunków, a niniejszy program próbuje symulować jej proces w celu rozwiązania problemu komiwojażera.
Na potrzeby rozwiązania problemu definiuje się populacje, która w tym przypadku  jest zbiorem możliwych rozwiązań problemu
(jest jedną z poprawnych ścieżek).
Na danej populacji przeprowadzamy operacje symulujące jej ewolucję w konsekwencji dochodząc do coraz to lepszych rezultatów. Taki wynik otrzymywany jest
poprzez zdefiniowanie odpowiednich zasad i mechanizmów zachodzących w czasie tej symulacji. Głownymi operacjami wykonywanymi w celu symulacji zjawiska
ewolucji są:

\par \textbf{Rozmnażanie}
(krzyżowanie) w którym bierze udział dwoje rodziców, tj. rozwiązań z poprzedniej generacji. Wynikiem tej operacji jest potomek, tj. nowe rozwiązanie,
które poddane zostaje ocenie pod względem przystosowania do środowiska w etapie selekcji (więcej informacji w punktach poniżej).

\par \textbf{Mutacje},
które polegają na wykonaniu losowych "przekłamań" w rozwiązaniu zaraz po utworzeniu nowego osobnika (rozwiązania).
Efekt ten zachodzi z zadanym prawdopodobieństem i ma za zadanie wprowadzić zróżnicowanie do póli rozwiązań, tym samym pozytywnie oddziałowując na
problem związany z "zakleszczeniem" się w obrębie lokalnego minimum.

\par \textbf{Selekcja},
polegająca na ocenie przystosowania danego osobnika, w tym przypadku liczony jest koszt tego rozwiązania.
Dla tego konkretnego problemu niższy koszt rozwiązania oznacza lepsze rozwiązanie i tym samym lepsze przystosowanie.
Tak ocenione rozwiązanie wzbogaca póle gatunku i obejmuje pozycje osobnika (rozwiązania), który może zostać poddany kolejnym krzyżowaniom.

\subsection{Implementacja - rozmnażanie}
Rozmnażanie (ang. crossing) zostało zaimplementawane na dwa sposoby. Pierwszym z nich jest PMX Crossover (Partially-mapped crossover).

\par \textbf{PMX Crossover}
\par Polega on na losowym
wyborze zakresu rozwiązania jednego z rodziców oraz podstawieniu go w to samo miejsce u dziecka. Reszta elementów w genotypie dziecka powinna zostać 
wypłeniona elementami drugiego rodzica. Takie wypłenienie jednak może spodowować powtarzające sie wierzchołki i dla tego problemu jest nie dopuszczalne.
W celu naprawy rozwiązania ( pozbycia sie podwójnie występującuch wierzchołków) wierzchołki pobrane od drugiego z rodziców, jeżeli występują dwuktornie
powinny zostać dopowiednio zamienione, tak, aby w ich miejscach pojawiły się wierzchołki dotychczas nie znajdujące sie w genotypie dziecka.
Implementacja przedstawiona przez autora osiąga ten cel bez dodatkowych alokacji, przeprowadzając wszytkie wymagane poprawki "w miejscu", od razu na
tablicy wierzchołków nowego potomka. Decyzja na implementację takiego wariantu rozwiązania została podjęta ze względu na relatywnie wolną implementację
dynamicznej alokacji pamieci używanej w standardowej bibliotece wykorzystywanego języka oraz (w praktyce) bardzo dobrą wydajność przy przeglądzie
małej wielkości tablic (zależnie od modelu procesora ta wartości może wachać się od kilkudziesięciu do nawet kilkuset elementów), nawet w przypadkach
dużo słabszej teorytycznej złożoności obliczeniowej.
Listing kodu z implementacją znajduje się poniżej.

\lstinputlisting[caption=Implementacja PMX crossover "w miejscu",language=C++]
{crossover-pmx.cc}

\par \textbf{OX Crossover}
\par Drugim z zastosowanych algorytmów krzyżowania jest OX Crossover, który podobnie jak poprzednik polega na losowym wyborze częsci pierwszego z rodziców,
oraz przekopiowaniem go do rozwiązania-potomka. Następnie, należy iterować, zaczynając od końca wylosowanego obszaru wymiany genów, aż dotrzemy do jego
początku. Aby było to możliwe należy zresetować iterator na początek sekwencji po dotarciu na jego koniec. W trakcie iterowania do w rozwiązaniu
potomnym należy umieszczać kolejne wierzchołi tylko jeżeli nie występują jeszcze u rozwiązania-potomka.
Listing kodu z implementacją znajduje się poniżej.

\lstinputlisting[caption=Implementacja OX crossover,language=C++]
{crossover-ox.cc}

\section{Algorytm mrówkowy}
\par Algoytm mrówkowy jest metaheurystycznym algorytmem stadnym, który powstał jako wynik obserwacji i chęci naśladowania zachowania mrówek, które
używając zjawiska stygmergii potrafia przekazywać między sobą informację o drodze którą przemierzają w celu zdobycia pokarmu, dzięki temu są w stanie
współpracować i osiągać dużo lepsze rezultaty. Stworzenia te, jak się okazuje przemieszczając się, na odwiedzonych ścieżkach pozostawiają parującą
substancję zwaną feromonami. Feromony pozostawione przez poprzednie osobniki kroczące daną scieżką są wykrywane przez kolejne i odczytywane są
jako wskazówka dotycząca kierunku w którym mrówka powinna się kierować. Takie zachowanie z czasem skutkuje samoistnym wyznaczeniem najszybszej ścieżki
do celu. Dzieję się tak prawdopodobnie z tego powodu, że mrówki podróżując w stronę pożywienia będą częsciej i gęsciej oznacza te scieżki, które są
najkrótsze, ze względu na to że nateżenie ruchu bedzie na nich największe ze względu na szybszą podróż względem pozostałych. Ścieżki najczęsciej odwiedzane
będą pokryte największa ilością feromonu, a więc inne osobniki z łatwością będą wybierały właśnie tę scieżke i tym samym potęgowały dodatkowo ten efekt,
gdyż same też wydzielać będą dodatkowe ilości feromonu. W celu zasymulowania tego zachowania autor przedstawił rozszerzyć macierz sąsiedztwa o drógą,
oddzielną macierz o takiej samej wielkości i strukturze, ale przechowującej ilość posiadanego feromonu dla każdej ze scieżek, to jest każdej dwu
elementowej pary wierzchołków. Z podobieństwa tych dwóch struktur wynika bezpośrednio, że ilość feromonu różni się, zależnie od tego w którą stronę
rozpatrujemy połączenie między wierzchołkami - ilość feromony nie musi być równa gdy rozpatrujemy scieżkę z miasta A do miasta B oraz z miasta B do A.
Do przechowywania informacji o każdej z mrówek, autor postanowił zastosować strukturę, która przechowuje informację o dotychczasowej drodze przebytej
przez mrówkę, oraz jej dotychczasowy koszt (aby nie wyliczać go za każdym razem na nowo gdy dołączany jest kolejny wierzchołek a jedynie zaktualizować
ten koszt o wartość od aktualnego do następnego wierzchołka). Dodatkowo przechowywana jest referencja do macierzy sąsiedztwa oraz tej z poziomami
feromony dla każdej z scieżki, aby ułatwić szybkie sprawdzanie kosztu krawędzi po której podróżują mrówki oraz, aby mogły aktualizować stan feromonu
przy wykonywaniu kolejnych ruchów z wierzchołka do wierzchołka. Dodatkowo autor postanowił wydzielić operację ruchu, oraz aktualizacji poziomu feromonu
do osobnej abstrakcji, aby wygodnie zaimplementować różne warianty algorytmu, o których więcej później.

\lstinputlisting[caption=Implementacja klasy przechowującej informację o mrówce,language=C++]
{ant-impl.cc}

Klasa bazowa, której implementacja widnieje na poprzednim listingu. Jej implementacja została pozbawiona kilku drobnych szczegółów w celu poprawienia
jej czytelności. To co jest jednak ważne, to trzy metody które ona implementuje.
Pierwszą z nich jest metoda resetowania (życia/scieżki) mrówki. Kiedy mrówka przejdzie po tylu wierzchołkach, że zapełni swoją przebytą scieżkę, tj.
jej rozmiar zrówna się z rozmiarem problemu, wówczas uzyskana scieżka sprawdzana jest pod kątem atrakcyjności względem dotychczas zapisanego
najlepszego rozwiązania i jeżeli osiąga lepszy (niższy) koszt, zajmuje jego miejsce. Po tym, scieżka mrówki zostaje zresetowana i zainicjalizowana
losowo wybranym wierzchołkiem początkowym, od którego cały cykl rozpocznie się na nowo.
Kolejne dwie metody są ze sobą scieśle powiązane, dlatego zostaną omwione wspólnie. Mowa o metodze decide oraz value, Druga z nich odpowiedzialna
jest za ocene ścieżki (przynaniu jej konkretnej wartości), zależnie od ilości posiadanego feromonu oraz jej kosztu. Wartości scieżki obliczane jest 
za pomocą równiania przedstawionego na listingu ponizej. Warto, jeszcze dodać, że w przypadku zerowego kosztu scieżki zwracana jest arbitralnie wybrana
"bardzo wysoka" wartość, która dzięki temu faworyzuje te ścieżki. Wartość nie może być jednak zbyt duża, aby nie powodować zakleszczenia się w minimum
loklanym poprzez ślepe podąrzanie jedynie za scieżkami o zerowym koszcie, gdyż w efekcie prowadzić mogą do bardzo kosztownej drogi w ostateczności.

\lstinputlisting[caption=Implementacja podstawowych operacji - reset oraz value,language=C++]
{ant-baseop.cc}

Ostanią już ze wspomnianych funkcji klasy bazowej jest funkcja odpowiedzialna za podjęcie decyzji o tym która drogę ostatecznie wybrać gdy przyjdzie
czas wykonania ruchu do kolejnego miasta. Tą funkcją jest funkcja decide. Wybiera ona scieżkę z zadanym prawdopodobieństwem dodatkowo bazując na wartości,
którą otrzymała od przedstawionej wcześniej funkcji value. Dodatkowo, obligatoryjny jest jeszcze warunek nieobecności na dotychczasowej przebytej ścieżce,
z oczywistych względów. Implementacja funkcji decide znajduje sie na listingu poniżej.

\lstinputlisting[caption=Implementacja podstawowych operacji - decide,language=C++]
{ant-decide.cc}

Pozostaje jeszcze zdefiniować procedurę związana z wykonaniem kroku przez mrówkę. Ich implementacja różni się w zależności od zastosowanej strategii
dystybucji feromonu. Aby wygodnie wydzielić wspomnianą abtrakcję autor zdecydował się na przedstawienie dodatkowej klasy, która w odpowiedni sposób
implementuje potrzebną metodę poruszania sie po grafie. Jej implementacja znajduje się na listingu poniżej.

\lstinputlisting[caption=Wydzielenie operacji ruchu do osobnej abstrakcji,language=C++]
{ant-spec.cc}

Implementacja konkretnej strategi ruchu osiągana jest poprzez jawną specjalizacje szablonu, tak jak pokazano na listingu poniżej.

\lstinputlisting[caption=Jawna specjalizacja szablonu dla każdej z strategii,language=C++]
{ant-move.cc}

W miejscu użycia obiektu this->phro, intencja programisty może nie być do końca oczywista, dlatego autor zobiązany jest udzielić pewnego wyjaśnienia.
Wspomniany obiekt jest opsiywaną wcześniej macierzą (podobną do sąsiedztwa), która przechowuje informację o ilości zgromadzonego feromonu na wskazywanej
ścieżce. Metoda set ustawia wskazaną ilość feromonu na wskazanej scieżce. W przypadku strategii DAS użyta została stała (zdefiniowana w zasięgu głównej
klasy) "phro\_place\_amnt" i występuje w postaci parametru który w czasie testu dobierany empirycznie, tak aby dawać jak najlepsze wyniki.

\pagebreak
\section{Eksperymenty obliczeniowe}
Czas wykonywanych operacji był mierzony na systemie opisanym w wprowadzeniu.
Do pomiaru czasu wykorzystano standardową bibliotekę języka C++, std::chrono.
Pomiar czasu wykonywany był w nanosekundach. Każdy pomiar wykonywany był 50 razy i ostatecznym wynikiem była średnia czasu z wszystkich prób.
Dodatkowo przed dokonywaniem takiego pomiaru uruchamiano 10 razy przebieg mierzonego kawałka kodu bez mierzenia jego czasu po to aby wytrenować układy przewidywania odwoływań do pamięci,
tj. branch prediction, cache prefetching.
Dodatkowo każdy z algorytmów przed pomiarem został dostrojony (za pomocą paramterów specyficznych dla każdego algorytmu opisanych we wcześniejszych
paragrafach) tak aby czas jego wykonania nie odbiegał znacznie od innych algorytmów. Konsekwencją tego są czasy wykonania bardzo zbliżone do siebie,
co jest powodem pominięcia przez autora analizy złożoności czasowej. Omówione natomiast zostały ich rezultaty, tj. porównanie wygenerowanej
scieżki z najlepszą możliwa dla danego zestawu danych.
Poniżej znajduje się kod realizujący mierzenie czasu.

\lstinputlisting[caption=Funkcje pomocnicze do liczenia czasu wykonania algorytmów,language=c++]
{timeutils.hpp}

\par
Poniżej znajdują się wyniki pomiarów zgromadzone dla różnych wariantów każdego z algorytmów.

\begin{center}
\begin{tabularx}{1.0\textwidth} {
	| >{\centering\arraybackslash}X
	| >{\centering\arraybackslash}X
	| >{\centering\arraybackslash}X
	| >{\centering\arraybackslash}X
	| >{\centering\arraybackslash}X | }
	\hline
	\multicolumn{5}{|c|}{Algorytm mrówkowy - wyniki ekperymentu 1} \\
	\hline
	Wielkość problemu & Czas wykonania (ms) & Błąd procentowy (w przybliżeniu) & Ilość mrówek & Liczba iteracji \\
	\hline
	17 & 112 & 0 &100 &50 \\
	\hline
	34 & 352 & 0 &100 &50 \\
	\hline
	36 & 398 & 0 &100 &50 \\
	\hline
	39 & 458 & 0 &100 &50 \\
	\hline
	43 & 496 & 1 &100 &50 \\
	\hline
	45 & 596 & 2 &100 &50 \\
	\hline
	48 & 716 & 5 &100 &50 \\
	\hline
	53 & 804 & 5 &100 &50 \\
	\hline
	56 & 916 & 4 &100 &50 \\
	\hline
	65 & 1230 & 8 &100 &50 \\
	\hline
	70 & 1444 & 6 &100 &50 \\
	\hline
	71 & 1410 & 11 &100 &50 \\
	\hline
	100 & 2916 & 12 &100 &50 \\
	\hline
	17 & 162 & 0 &150 &50 \\
	\hline
	34 & 522 & 0 &150 &50 \\
	\hline
	36 & 590 & 0 &150 &50 \\
	\hline
	39 & 682 & 0 &150 &50 \\
	\hline
	43 & 740 & 2 &150 &50 \\
	\hline
	45 & 890 & 1 &150 &50 \\
	\hline
	48 & 1076 & 4 &150 &50 \\
	\hline
	53 & 1196 & 4 &150 &50 \\
	\hline
	56 & 1364 & 4 &150 &50 \\
	\hline
	65 & 1844 & 6 &150 &50 \\
	\hline
	70 & 2174 & 8 &150 &50 \\
	\hline
	71 & 2108 & 9 &150 &50 \\
	\hline
	100 & 4376 & 10 &150 &50 \\
	\hline
	17 & 214 & 0 &200 &50 \\
	\hline
	34 & 696 & 0 &200 &50 \\
	\hline
	36 & 774 & 0 &200 &50 \\
	\hline
	39 & 908 & 0 &200 &50 \\
	\hline
	43 & 986 & 3 &200 &50 \\
	\hline
	45 & 1176 & 3 &200 &50 \\
	\hline
	48 & 1430 & 5 &200 &50 \\
	\hline
	53 & 1582 & 4 &200 &50 \\
	\hline
	56 & 1814 & 7 &200 &50 \\
	\hline
	65 & 2442 & 10 &200 &50 \\
	\hline
	70 & 2912 & 7 &200 &50 \\
	\hline
	71 & 2798 & 9 &200 &50 \\
	\hline
	100 & 5844 & 9 &200 &50 \\
	\hline
\end{tabularx}

\begin{tabularx}{1.0\textwidth} {
	| >{\centering\arraybackslash}X
	| >{\centering\arraybackslash}X
	| >{\centering\arraybackslash}X
	| >{\centering\arraybackslash}X
	| >{\centering\arraybackslash}X | }
	\hline
	\multicolumn{5}{|c|}{Algorytm mrówkowy- wyniki eksperymentu 2} \\
	\hline
	Wielkość problemu & Czas wykonania (ms) & Błąd procentowy (w przybliżeniu)& Ilość mrówek & Liczba iteracji \\
	\hline
	17 & 110 & 0 &50 &100 \\
	\hline
	34 & 352 & 0 &50 &100 \\
	\hline
	36 & 390 & 0 &50 &100 \\
	\hline
	39 & 460 & 0 &50 &100 \\
	\hline
	43 & 502 & 4 &50 &100 \\
	\hline
	45 & 600 & 7 &50 &100 \\
	\hline
	48 & 710 & 6 &50 &100 \\
	\hline
	53 & 800 & 8 &50 &100 \\
	\hline
	56 & 914 & 9 &50 &100 \\
	\hline
	65 & 1230 & 10 &50 &100 \\
	\hline
	70 & 1436 & 8 &50 &100 \\
	\hline
	71 & 1388 & 10 &50 &100 \\
	\hline
	100 & 2878 & 17 &50 &100 \\
	\hline
	17 & 162 & 0 &50 &150 \\
	\hline
	34 & 534 & 0 &50 &150 \\
	\hline
	36 & 588 & 0 &50 &150 \\
	\hline
	39 & 682 & 0 &50 &150 \\
	\hline
	43 & 750 & 1 &50 &150 \\
	\hline
	45 & 888 & 3 &50 &150 \\
	\hline
	48 & 1070 & 6 &50 &150 \\
	\hline
	53 & 1178 & 4 &50 &150 \\
	\hline
	56 & 1350 & 4 &50 &150 \\
	\hline
	65 & 1826 & 8 &50 &150 \\
	\hline
	70 & 2150 & 9 &50 &150 \\
	\hline
	71 & 2070 & 10 &50 &150 \\
	\hline
	100 & 4304 & 15 &50 &150 \\
	\hline
	17 & 224 & 0 &50 &200 \\
	\hline
	34 & 706 & 0 &50 &200 \\
	\hline
	36 & 782 & 0 &50 &200 \\
	\hline
	39 & 910 & 0 &50 &200 \\
	\hline
	43 & 1016 & 1 &50 &200 \\
	\hline
	45 & 1170 & 2 &50 &200 \\
	\hline
	48 & 1420 & 3 &50 &200 \\
	\hline
	53 & 1562 & 3 &50 &200 \\
	\hline
	56 & 1802 & 2 &50 &200 \\
	\hline
	65 & 2408 & 7 &50 &200 \\
	\hline
	70 & 2852 & 8 &50 &200 \\
	\hline
	71 & 2730 & 7 &50 &200 \\
	\hline
	100 & 5706 & 8 &50 &200 \\
	\hline
\end{tabularx}

\bigskip

\begin{tabularx}{1.0\textwidth} {
	| >{\centering\arraybackslash}X
	| >{\centering\arraybackslash}X
	| >{\centering\arraybackslash}X
	| >{\centering\arraybackslash}X | }
	\hline
	\multicolumn{4}{|c|}{Algorytm genetyczny - wyniki eksperymentu dla krzyżowania ox} \\
	\hline
	wielkość problemu & czas wykonania (ms) & błąd procentowy (w przybliżeniu) & wielkość populacji \\
	\hline
	17  & 302   & 0   & 25 \\
	\hline
	34  & 902   & 0   & 25 \\
	\hline
	36  & 1004  & 0   & 25 \\
	\hline
	39  & 1176  & 0   & 25 \\
	\hline
	43  & 1448  & 0   & 25 \\
	\hline
	45  & 1566  & 0   & 25 \\
	\hline
	48  & 1834  & 2   & 25 \\
	\hline
	53  & 2236  & 3   & 25 \\
	\hline
	56  & 2572  & 2   & 25 \\
	\hline
	65  & 3810  & 6   & 25 \\
	\hline
	70  & 4394  & 5   & 25 \\
	\hline
	71  & 4548  & 7   & 25 \\
	\hline
	100 & 10878 & 13  & 25 \\
	\hline

	17  & 1582  & 0   & 50 \\
	\hline
	34  & 4318  & 0   & 50 \\
	\hline
	36  & 4790  & 0   & 50 \\
	\hline
	39  & 5544  & 0   & 50 \\
	\hline
	43  & 6726  & 0   & 50 \\
	\hline
	45  & 7122  & 0   & 50 \\
	\hline
	48  & 8102  & 0   & 50 \\
	\hline
	53  & 9910  & 1   & 50 \\
	\hline
	56  & 11290 & 0   & 50 \\
	\hline
	65  & 15520 & 4   & 50 \\
	\hline
	70  & 18750 & 2   & 50 \\
	\hline
	71  & 19340 & 6   & 50 \\
	\hline
	100 & 45292 & 9   & 50 \\
	\hline
\end{tabularx}

\begin{tabularx}{1.0\textwidth} {
	| >{\centering\arraybackslash}X
	| >{\centering\arraybackslash}X
	| >{\centering\arraybackslash}X
	| >{\centering\arraybackslash}X | }
	\hline
	\multicolumn{4}{|c|}{Algorytm genetyczny - wyniki eksperymentu dla krzyżowania pmx} \\
	\hline
	wielkość problemu & czas wykonania (ms) & błąd procentowy (w przybliżeniu) & wielkość populacji \\
	\hline
	17  & 210   & 0   & 25 \\
	\hline
	34  & 496   & 0   & 25 \\
	\hline
	36  & 522   & 0   & 25 \\
	\hline
	39  & 606   & 0   & 25 \\
	\hline
	43  & 660   & 1   & 25 \\
	\hline
	45  & 702   & 2   & 25 \\
	\hline
	48  & 778   & 2   & 25 \\
	\hline
	53  & 862   & 5   & 25 \\
	\hline
	56  & 912   & 5   & 25 \\
	\hline
	65  & 1106  & 9   & 25 \\
	\hline
	70  & 1238  & 12  & 25 \\
	\hline
	71  & 1256  & 11  & 25 \\
	\hline
	100 & 1982  & 11  & 25 \\

	\hline
	17  & 964   & 0 & 50 \\
	\hline
	34  & 2352  & 0 & 50 \\
	\hline
	36  & 2474  & 0 & 50 \\
	\hline
	39  & 2840  & 0 & 50 \\
	\hline
	43  & 3158  & 0 & 50 \\
	\hline
	45  & 3422  & 1 & 50 \\
	\hline
	48  & 3524  & 1 & 50 \\
	\hline
	53  & 4086  & 3 & 50 \\
	\hline
	56  & 4224  & 2 & 50 \\
	\hline
	65  & 5206  & 4 & 50 \\
	\hline
	70  & 5872  & 7 & 50 \\
	\hline
	71  & 5938  & 10 & 50 \\
	\hline
	100 & 9458  & 10 & 50 \\
	\hline
\end{tabularx}

\end{center}

\newpage

\section{Wnioski}
\hspace{\parindent}
\par Algorytm genetyczny, mimo polegania w dużym stopniu na procesach losowych daje dobre rezultaty. Strategią krzyżowania, która najlepiej sprawdziła
się w tym przypadku jest OX Crossover w połączeniu z selekcja w trybie rankingowym. Bardzo ważnym elementem algorytmu okazało się wymiana aktualnej
populacji mimo otrzymywania chilowo gorszych wyników z jednoczesnym zachowaniem częsci najlepszej populacji z generacji poprzedniej. Implementacja wcześniej
testowana przez autora, której brakowało wspomnianej cechy bardzo szybko zapełniała pulę rozwiązań podobnymi do siebie rozwiązaniami, a przez to cały algorytm
grzązł w lokalnym minimum. Promowanie zróżnicowania w puli zdecydowanie miała pozytywny wpływ na wynik całego algorytmu. Prostota implementacji
tego algorytmu oraz jego efektywność zdecydowanie umieszcza go na pozycji aglorytmu który powinien być rozważonany przy rozwiązywaniu innych problemów.

\par Algorytm mrówkowy, również w dużej mierze polega na elementach losowości, jednakże tak jak w przypadku poprzedniego algorytmu dostacza relatywnie dobrych rozwiązań.
Ciężko jest stwierdzić jednoznacznie który algorytm dostarcza lepszych rozwiązań. Autor pragnie jednak zaznaczyć, że dobranie parametrów algorytmu mrówkowego, tak aby uzyskiwać
średnio dobre rezultaty był ekstremalnie trudny. Ostatecznie autorowi udało ustalić się zestaw parametrów, odmienny dla różnych grup (podzielonych zależnie od wielkości problemu),
który dostatecznie zbliżył się do akceptowalnych rozwiązań. Nie wykluczone, że dalsze optymalizacje mogą zostać poczynione pod kątem konkretnego problemu i tym samym dać jeszcze lepsze
rezultaty, niż te które udało się uzykać autorowi.

\bibliographystyle{abbrv}
\bibliography{ref}
\end{document}

