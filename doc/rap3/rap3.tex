\documentclass[polish,polish,a4paper]{article}
\usepackage[T1]{fontenc}
\usepackage[utf8]{inputenc}
\usepackage{babel}
\usepackage{pslatex}
\usepackage{graphicx}
\usepackage{tikz}
\usepackage{pgfplots}
\usepackage{anysize}
\usepackage{pgfgantt}
\usepackage{tabularx}
\usepackage{float}
\usepackage{latexsym,amsmath}
\marginsize{2.5cm}{2.5cm}{3cm}{3cm}


\newcommand{\name}[1]{\sffamily\bfseries\scriptsize #1}

\newcommand{\frontpage}[8]{
%% #1 - nazwa kursu
%% #2 - kierunek 
%% #3 - termin 
%% #4 - temat 
%% #5 - problem
%% #6 - data

\vspace{2cm}

\begin{tabular}{|p{0.72\textwidth}|p{0.28\textwidth}|}
\hline
\multicolumn{2}{|c|}{}\\
\multicolumn{2}{|c|}{{\LARGE #1}}\\
\multicolumn{2}{|c|}{}\\
\hline
\name{Kierunek} & \name{Termin}\\
\multicolumn{1}{|c|}{\textit{#2}} & \multicolumn{1}{|c|}{\textit{#3}} \\
\hline
\name{Temat} & \name{Problem}\\
\multicolumn{1}{|c|}{\textit{#4}} & \multicolumn{1}{|c|}{\textit{#5}} \\
\hline
\name{Prowadzący} & \name{data}\\
\multicolumn{1}{|c|}{\textit{mgr inż. Radosław Idzikowski}} & \multicolumn{1}{|c|}{\textit{#6}} \\
\hline
\end{tabular}

}

\usepackage{listings}
\usepackage{xcolor} % for setting colors


% set the default code style
\lstset{ % General setup for the package
	basicstyle=\small,
	numbers=left,
	frame=tb,
	tabsize=2,
	columns=fixed,
	showstringspaces=false,
	showtabs=false,
	keepspaces,
	commentstyle=\color{red},
	keywordstyle=\color{blue}
}

\title{Sprawozdanie SPD}
\begin{document}
% #1 - nazwa kursu #2 - kierunek  #3 - termin #4 - temat #5 - problem #6 - data
\frontpage{Projektowanie Efektywnych Algorytmów}{Informatyka}{Czwartek 19:05}{Przeszukiwanie tabu oraz simulowanie wyżarzanie}{TSP}{\today}
\pagestyle{empty}
\newpage

\section{Opis problemu}

Problem komiwojażera (ang. travelling salesman problem, TSP) jest zagadnieniem optymalizacyjnym, polegającym na znalezieniu minimalnego cyklu Hamiltona w pełnym grafie ważonym. Do rozwiązania problemu tym razem autor posłużył się dwoma algorytmami aproksymacyjnymi o których szerzej w dalszej częsci.
Do implementacji rozwiązania posłużył język C++, gdyż zdaniem autora pozwala on na wygodne opisywanie wysokopoziomowej abstrakcji oraz dostępne kompilatory
potrafią generować efektywny kod. Testy przeprowadzane były na systemie Ubuntu Linux 19.10 64bit z kompilatorem clang 9.0 na procesorze Intel i5-3570K (4x3.8GHz) z 8GB RAM.

\section{Struktury danych - sposób przechowywania informacji}

\subsection{Graf}

Problem komiwojażera można zdefiniować dla różnej ilości wieszchołków, dalej nazywaną wielkością problemu oraz różnych wag poszczególnych dróg łączących te wierzchołki.
Z tego względu wymaganym mechanizmem wczytywania informacji o problemie będzie plik, który zawiera wielkość grafu oraz wagi poszczególnych wierzchołków.
Wagi podawane są w postaci macierzy, w której kolejne elementy w wierszu oddzielane są znakami białymi, a same wiersze oddzielane są znakiem nowej lini.
Tak zapisane dane należy sparsować i zapisać w postaci pozwalającej na jak najprostszy oraz najszybszy sposób odczytu, ponieważ będzie ona intensywnie wykorzystywana przy rozwiązywaniu problemu.
Autor zdecydował się na przechowywanie danych w postaci kwadratowej macierzy sąsiedztwa o rozmiarze równym ilości wierzchołków.
I do jej reprezentacji została wykorzystana pojedyncza tablica typów int64\_t o rozmiarze $N^{2}$, gdzie N to rozmiar problemu.
Poniżej znajduje się definicja klasy MSTMatrix która implementuje przechowywanie macierzy sąsiedztwa.

\lstinputlisting[caption=Struktura danych MSTMatrix,language=C++]
{mst.hpp}

Jeden drobny szczegół pozostaje nie wyjaśniony - typ Edge, który klasa MSTMatrix przyjmuje jako argument przy metodzie dodawania krawędzi do grafu.
Jej implementacja jest prosta i sprowadza się do przetrzymywania informacji numeru wierzchołka źródłowego i docelowego, oraz wagi opisywanego połączenia.
Metoda MSTMatrix::add korzysta z jej pól, aby odpowiednio wypełnić informacje we własnej macierzy sąsiedztwa.
Rozwiązanie problemu będzie zwracane jako typ Path ktory jest lekkim wrapperem na klase std::vector z standardowej biblioteki.
Jego implementacja znajduje sie poniżej.

\lstinputlisting[caption=Struktura danych Path,language=C++]
{path.hpp}

Został jeszcze jeden element który wymaga krótkiego wyjaśnienia. Jest to funkcja kosztu scieżki która zwraca koszt danej scieżki na podstawie argumentu w postaci macierzy sąsiedztwa
oraz sekwencji kolejno odwiedzanych wierzchołków.

\section{Algorytm genetyczny}
\par Algorytmy genetyczne polegają na wykorzystują mechanizmy występujące w przyrodzie, a dokładniej próbuje symulować proces ewolucji biologicznej.
Na potrzeby rozwiązania problemu definiuje się populacje, która w jest zbiorem możliwych rozwiązań problemów ( w tym przypadku jedną z poprawnych ścieżek).
Na danej populacji przeprowadzamy operacje symulujące jej ewolucję w konsekwencji dochodząc do coraz to lepszych rezultatów. Taki wynik otrzymywany jest
poprzez zdefiniowanie odpowiednich zasad i mechanizmów zachodzących w czasie tej symulacji. Głownycmi operacjami wykonywanymi w celu symulacji zjawiska
ewolucji sa:

\par \textbf{Rozmnażanie}
(krzyżowanie) w którym bierze przypadku bierze udział dwoje rodziców, rozwiązań z poprzedniej generacji. Wynikiem tej operacji jest nowe rozwiązanie,
które dodane jest do aktualnej póli gatunku.

\par \textbf{Mutacje},
które polegają na wykonaniu losowych "przekłamań" w rozwiązaniu zaraz po utworzeniu nowego osobnika (rozwiązania).
Efekt ten zachodzi z zadanych prawdopodobieństem.

\par \textbf{Selekcja},
polegająca na ocenie przystosowania danego osobnika (tutaj koszt scieżki - im mniejszy koszt tym lepiej przystosowany osobnik)

\subsection{Implementacja - rozmnażanie}
Rozmnażanie (ang. crossing) zostało zaimplementawane na dwa sposoby. Pierwszym z nich jest PMX Crossover (Partially-mapped crossover).

\par \textbf{PMX Crossover}
\par Polega on na losowym
wyborze zakresu rozwiązania jednego z rodziców oraz podstawieniu go w to samo miejsce u dziecka. Reszta elementów w genotypie dziecka powinna zostać 
wypłeniona elementami drugiego rodzica. Takie wypłenienie jednak może spodowować powtarzające sie wierzchołki i dla tego problemu jest nie dopuszczalne.
W celu naprawy rozwiązania ( pozbycia sie podwójnie występującuch wierzchołków) wierzchołki pobrane od drugiego z rodziców, jeżeli występują dwuktornie
powinny zostać dopowiednio zamienione, tak, aby w ich miejscach pojawiły się wierzchołki dotychczas nie znajdujące sie w genotypie dziecka.
Listing kodu z implementacją znajduje się poniżej.

\pagebreak

\lstinputlisting[caption=Struktura danych Path,language=C++]
{crossover-pmx.cc}

\pagebreak

\par \textbf{OX Crossover}
\par Drugim z zastosowanych algorytmów krzyżowania jest OX Crossover, który podobnie jak poprzednik polega na losowym wyborze częsci pierwszego z rodziców,
oraz przekopiowaniem go do rozwiązania-dziecka. Następnie zaczynając terować w rodzicu drugim od końca przekopiownego obszaru do jego początku
(w trakcie iteracji jeżeli napotkamy koniec, zaczynamy od samego początku) i kolejno wstawiamy wierzchołki które jescze nie wystąpiły.
Listing kodu z implementacją znajduje się poniżej.

\lstinputlisting[caption=Struktura danych Path,language=C++]
{crossover-ox.cc}


\pagebreak
\section{Eksperymenty obliczeniowe}
Czas wykonywanych operacji był mierzony na systemie opisanym w wprowadzeniu.
Do pomiaru czasu wykorzystano standardową bibliotekę języka C++, std::chrono.
Pomiar czasu wykonywany był w nanosekundach. Każdy pomiar wykonywany był 50 razy i ostatecznym wynikiem była średnia czasu z wszystkich prób.
Dodatkowo przed dokonywaniem takiego pomiaru uruchamiano 10 razy przebieg mierzonego kawałka kodu bez mierzenia jego czasu po to aby wytrenować układy przewidywania odwoływań do pamięci,
tj. branch prediction, cache prefetching.
Dodatkowo każdy z algorytmów przed pomiarem został dostrojony (za pomocą paramterów specyficznych dla każdego algorytmu opisanych we wcześniejszych
paragrafach) tak aby czas jego wykonania nie odbiegał znacznie od innych algorytmów. Konsekwencją tego są czasy wykonania bardzo zbliżone do siebie,
co jest powodem pominięcia przez autora analizy złożoności czasowej. Omówione natomiast zostały ich rezultaty, tj. porównanie wygenerowanej
scieżki z najlepszą możliwa dla danego zestawu danych.
Poniżej znajduje się kod realizujący mierzenie czasu.

\lstinputlisting[caption=Funkcje pomocnicze do liczenia czasu wykonania algorytmów,language=c++]
{timeutils.hpp}

\par
Poniżej znajdują się wyniki pomiarów zgromadzone dla różnych wariantów każdego z algorytmów.

\begin{center}

\begin{tabularx}{1.0\textwidth} {
	| >{\centering\arraybackslash}X
	| >{\centering\arraybackslash}X
	| >{\centering\arraybackslash}X
	| >{\centering\arraybackslash}X
	| >{\centering\arraybackslash}X | }
	\hline
	\multicolumn{5}{|c|}{Przeszukiwania tabu - metada zamiany (swap) - start losowy} \\
	\hline
	Wielkość problemu & Czas wykonania (ns) & Koszt znalezionej drogi & Koszt najlepszej drogi & błąd \\
	\hline
	17 & 4678691 & 2085 & 2085 & 0 \\
	\hline
	21 & 8122241 & 2707 & 2707 & 0 \\
	\hline
	24 & 11574596 & 1298 & 1272 & 26 \\
	\hline
	26 & 14488670 & 937 & 937 & 0 \\
	\hline
	29 & 19540981 & 1656 & 1610 & 46 \\
	\hline
	42 & 55657465 & 777 & 699 & 78 \\
	\hline
	58 & 140285446 & 30986 & 25395 & 5591 \\
	\hline
\end{tabularx}
\bigskip

\begin{tabularx}{1.0\textwidth} {
	| >{\centering\arraybackslash}X
	| >{\centering\arraybackslash}X
	| >{\centering\arraybackslash}X
	| >{\centering\arraybackslash}X
	| >{\centering\arraybackslash}X | }
	\hline
	\multicolumn{5}{|c|}{Przeszukiwania tabu - metada wstawiania (insert) - start losowy} \\
	\hline
	Wielkość problemu & Czas wykonania (ns) & Koszt znalezionej drogi & Koszt najlepszej drogi & błąd \\
	\hline
	17 & 5995145 & 3382 & 2085 & 1297 \\
	\hline
	21 & 10298361 & 4673 & 2707 & 1966 \\
	\hline
	24 & 14680920 & 3315 & 1272 & 2043 \\
	\hline
	26 & 18122428 & 2426 & 937 & 1489 \\
	\hline
	29 & 24541917 & 4855 & 1610 & 3245 \\
	\hline
	42 & 68625855 & 2934 & 699 & 2235 \\
	\hline
	58 & 179318868 & 113213 & 25395 & 87818 \\
	\hline
\end{tabularx}
\bigskip

\begin{tabularx}{1.0\textwidth} {
	| >{\centering\arraybackslash}X
	| >{\centering\arraybackslash}X
	| >{\centering\arraybackslash}X
	| >{\centering\arraybackslash}X
	| >{\centering\arraybackslash}X | }
	\hline
	\multicolumn{5}{|c|}{Przeszukiwania tabu - metada odwrócenia (reverse) - start losowy} \\
	\hline
	Wielkość problemu & Czas wykonania (ns) & Koszt znalezionej drogi & Koszt najlepszej drogi & błąd \\
	\hline
	17 & 5845171 & 2085 & 2085 & 0 \\
	\hline
	21 & 10171492 & 2707 & 2707 & 0 \\
	\hline
	24 & 14407409 & 1272 & 1272 & 0 \\
	\hline
	26 & 18198326 & 937 & 937 & 0 \\
	\hline
	29 & 24290646 & 1615 & 1610 & 5 \\
	\hline
	42 & 71764416 & 699 & 699 & 0 \\
	\hline
	58 & 169588431 & 25395 & 25395 & 0 \\
	\hline
\end{tabularx}
\bigskip

\begin{tabularx}{1.0\textwidth} {
	| >{\centering\arraybackslash}X
	| >{\centering\arraybackslash}X
	| >{\centering\arraybackslash}X
	| >{\centering\arraybackslash}X
	| >{\centering\arraybackslash}X | }
	\hline
	\multicolumn{5}{|c|}{Simulowane wyżarzanie - liniowa regulacja temperatury - start losowy} \\
	\hline
	Wielkość problemu & Czas wykonania (ns) & Koszt znalezionej drogi & Koszt najlepszej drogi & błąd \\
	\hline
	17 & 143268183& 2427 & 2085 & 342 \\
	\hline
	21 & 145967248& 3568 & 2707 & 861 \\
	\hline
	24 & 148627322& 1957 & 1272 & 685 \\
	\hline
	26 & 150143893& 1551 & 937 & 614 \\
	\hline
	29 & 153477613& 2508 & 1610 & 898 \\
	\hline
	42 & 163647317& 1897 & 699 & 1198 \\
	\hline
	58 & 177813088& 57551 & 25395 & 32156 \\
	\hline
\end{tabularx}
\bigskip

\begin{tabularx}{1.0\textwidth} {
	| >{\centering\arraybackslash}X
	| >{\centering\arraybackslash}X
	| >{\centering\arraybackslash}X
	| >{\centering\arraybackslash}X
	| >{\centering\arraybackslash}X | }
	\hline
	\multicolumn{5}{|c|}{Simulowane wyżarzanie - geometryczna regulacja temperatury - start losowy} \\
	\hline
	Wielkość problemu & Czas wykonania (ns) & Koszt znalezionej drogi & Koszt najlepszej drogi & błąd \\
	\hline
	17 & 150094480 & 2085  & 2085  & 0    \\
	\hline
	21 & 152922148 & 2757  & 2707  & 50   \\
	\hline
	24 & 155708966 & 1290  & 1272  & 18   \\
	\hline
	26 & 157297797 & 965   & 937   & 28   \\
	\hline
	29 & 160790359 & 1683  & 1610  & 73   \\
	\hline
	42 & 171444619 & 747   & 699   & 48   \\
	\hline
	58 & 186285346 & 28461 & 25395 & 3066 \\
	\hline
\end{tabularx}
\bigskip

\begin{tabularx}{1.0\textwidth} {
	| >{\centering\arraybackslash}X
	| >{\centering\arraybackslash}X
	| >{\centering\arraybackslash}X
	| >{\centering\arraybackslash}X
	| >{\centering\arraybackslash}X | }
	\hline
	\multicolumn{5}{|c|}{Simulowane wyżarzanie - logarytmiczna regulacja temperatury - start losowy} \\
	\hline
	Wielkość problemu & Czas wykonania (ns) & Koszt znalezionej drogi & Koszt najlepszej drogi & błąd \\
	\hline
	17 & 2090093 & 2085 & 2085 & 0 \\
	\hline
	21 & 2243627 & 2760 & 2707 & 53 \\
	\hline
	24 & 2072114 & 1330 & 1272 & 58 \\
	\hline
	26 & 2103146 & 1032 & 937 & 95 \\
	\hline
	29 & 2152118 & 1730 & 1610 & 120 \\
	\hline
	42 & 2366575 & 958 & 699 & 259 \\
	\hline
	58 & 2634296 & 27326 & 25395 & 1931 \\
	\hline
\end{tabularx}
\bigskip

\end{center}

\newpage

\newpage

\section{Wnioski}
\hspace{\parindent}
\par Algorytm genetyczny, mimo polegania w dużym stopniu na procesach losowych daje dobre rezultaty. Strategią krzyżowania, która najlepiej sprawdziła
się w tym przypadku jest OX Crossover w połączeniu z selekcja w trybie rankingowym. Bardzo ważnym elementem algorytmu okazało się wymiana aktualnej
populacji mimo otrzymywania chilowo gorszych wyników z jednoczesnym zachowaniem częsci najlepszej populacji z generacji poprzedniej. Prostota implementacji
tego algorytmu oraz jego efektywność zdecydowanie umieszcza go na pozycji algorytmów którego powinny być rozważone przy rozwiązywaniu innych problemów.

\bibliographystyle{abbrv}
\bibliography{ref}
\end{document}

